\documentclass[14pt,a4paper]{extarticle}

% -------------------------------
% БАЗОВІ ПАКЕТИ ТА МОВА
% -------------------------------
\usepackage{cmap} % Додано для коректного кодування шрифтів
\usepackage[ukrainian]{babel}
\usepackage[utf8]{inputenc}
\usepackage[T2A]{fontenc}

% -------------------------------
% ШРИФТ: Times New Roman (Повна підтримка кирилиці)
% -------------------------------
\usepackage{tempora}
\usepackage{newtxmath}

% -------------------------------
% ГЕОМЕТРІЯ СТОРІНКИ
% -------------------------------
\usepackage[
  left=1in,
  right=1in,
  top=1in,
  bottom=1in
]{geometry}

% -------------------------------
% МІЖРЯДКОВИЙ ІНТЕРВАЛ
% -------------------------------
\usepackage{setspace}
\setstretch{1.15}

% -------------------------------
% АБЗАЦИ
% -------------------------------
\setlength{\parindent}{0pt}
\setlength{\parskip}{6pt}

% -------------------------------
% ВИРІВНЮВАННЯ
% -------------------------------
\usepackage{microtype}

% -------------------------------
% ЗАГОЛОВКИ
% -------------------------------
\usepackage{titlesec}

\titleformat{\section}
  {\bfseries\large}
  {\thesection}
  {1em}
  {}

\titleformat{\subsection}
  {\bfseries\normalsize}
  {\thesubsection}
  {1em}
  {}

% -------------------------------
% МАТЕМАТИКА
% -------------------------------
\usepackage{amsmath, amssymb}

% -------------------------------
% ЗОБРАЖЕННЯ
% -------------------------------
\usepackage{graphicx}
\usepackage{caption}

\captionsetup{
  justification=centering,
  labelsep=space,
  name=рис.
}

% =====================================================
% ГІПЕРПОСИЛАННЯ ТА ІНТЕРАКТИВНІСТЬ
% =====================================================
\usepackage[hidelinks]{hyperref}
\urlstyle{same}

% =====================================================
% ЗМІСТ
% =====================================================
\usepackage{tocloft}
\setlength{\cftbeforesecskip}{6pt}

% -------------------------------
% ПОЧАТОК ДОКУМЕНТА
% -------------------------------
\begin{document}

% =====================================================
% ТИТУЛЬНИЙ ЛИСТ
% =====================================================
\begin{titlepage}
\begin{center}
МІНІСТЕРСТВО ОСВІТИ І НАУКИ УКРАЇНИ\\
\vspace{0.5em}
\textbf{НАЗВА УНІВЕРСИТЕТУ}\\
\vspace{0.5em}
Кафедра \rule{6cm}{0.4pt}

\vspace{5em}

\textbf{КУРСОВА РОБОТА}\\
\vspace{0.5em}
з дисципліни \rule{8cm}{0.4pt}

\vspace{3em}

\textbf{Тема:}\\
\vspace{0.5em}
\textbf{Розробка та дослідження узагальнених суфіксних дерев у багатопотоковому середовищі Node.js}

\vspace{5em}
\end{center}

\begin{flushright}
Виконав(ла): студент(ка) \rule{4cm}{0.4pt}\\
Група \rule{3cm}{0.4pt}

\vspace{1em}

Керівник: \rule{5cm}{0.4pt}
\end{flushright}

\vfill
\begin{center}
Київ -- 2026
\end{center}
\end{titlepage}

% =====================================================
% ЗМІСТ
% =====================================================
\tableofcontents
\newpage

% =====================================================
% ВСТУП
% =====================================================
\section*{ВСТУП}
\addcontentsline{toc}{section}{ВСТУП}

\newpage

% =====================================================
% РОЗДІЛ 1
% =====================================================
\section{ТЕОРЕТИЧНІ ОСНОВИ ТА ОГЛЯД ТЕХНОЛОГІЙ}

\subsection{Суфіксні дерева та генералізовані суфіксні дерева}

\textbf{Визначення}. Суфiксне дерево – це орiєнтоване дерево, що
iндексує всi суфiкси рядка $S$ довжини $n$. Кожна внутрiшня нода такого дерева, окрiм кореневої ноди, має принаймнi двi дитини. Ребра суфiксного дерева маркуються непорожнiм пiдрядком $S$, при чому жоднi два ребра не можуть мати мiтки, що почнаються з одного й того ж символу. Кiлькiсть листiв повинна точно дорiвнювати кiлькостi суфiксiв у заданому рядку. Ключовою властивiстю суфiксного дерева є те, що конкатенацiя ребер вiд кореневої ноди до листка вiдповiдає суфiксу в $S$, а асоцiйований з цим листком iндекс i вказує на початкову позицiю суфiкса $S_i$ .

Надамо більш формальне визначення вхідному рядку: 
\[
S = s_0, s_1, \ldots, s_{n-1}, \quad s_i \in \Sigma, \quad s_i \neq \$.
\]
Термінальний символ рядка, позначений символом $\$$, є унікальним і гарантує, що жоден суфікс не є префіксом жодного суфікса (властивість вільного префікса). Надалі в цій роботі будь-який рядок $S' = S$ вважається завершеним термінальним символом, навіть якщо він не зазначається явно.

Наприклад, для проіндесованого рядка $S = banana\$$ (рис. ~\ref{fig:png-1}), його суфіксами є $\$$, $a\$$, $na\$$, $ana\$$, $nana\$$, $anana\$$ і $banana\$$. Шляхом конкатенації міток ребер суфіксного дерева для S (рис. ~\ref{fig:png-2}) від кореневої ноди до 3-індексованого листка, отримуємо підрядок $S_3=ana\$$. Або ж, шляхом від кореневої ноди до 0-індексованого рядка – вхідний рядок $S_0=banana\$$.

\begin{figure}[h]
\centering
\includegraphics[width=0.7\textwidth]{images/рис. 1.png}
\caption{}
\label{fig:png-1}
\end{figure}

\begin{figure}[h]
\centering
\includegraphics[width=\textwidth]{images/рис. 2.png}
\caption{}
\label{fig:png-2}
\end{figure}

З іншого боку, суфіксне дерево, побудоване з одного рядка, є частковим випадком узагальненого суфіксного дерева. 

\textbf{Визначення}. Узагальненим суфіксним деревом називають орієнтоване дерево, що індексує всі суфікси для набору рядків T довжиною k .

Для побудови узагального суфіксного дерева усі k рядків з набору $T=\{S_1,S_2,…〖,S_k\$\}$ зливаються з використанням розмежувального символу # ($s_i  ≠#$), утворюючи суперрядок вигляду $S_1#S_2 \ldots S_K$.  Як і термінальний, розмежовувальний символ $#$  є унікальним,  і головна його функція полягає в запобіганні формуванню некоректних вхождень суфіксів, отриманих на перетині двох рядків з $T$. Для індексації суфіксів в узагальненому дереві використовується пара ($i:j$), де $i$ – номер рядка у $T$, $j$ – глобальний індекс суфікса в суперрядку. 

Доведемо на прикладі важливість злиття рядків з використанням розмежувального символу. Нехай $T=\{S_1,S_2\}$, де $S_1="Hello\$"$, $S_2="world\$"$.  В результаті прямого злиття заданих рядків $S_1$ і $S_2$ отримаємо суперрядок $S_m="Helloworld\$"$. Сканування ковзаючим вікном (TODO: референс на секцію) виявить підрядок $"low"$, який є підрядком $S_m$, але не є підрядком жодного з вхідних рядків $S_1$ або $S_2$. Врахування таких хибних підрядків приведе до побудови некоректного узагальненого суфіксного дерева, тому вважаємо нехтування розмежовувальним символом неприйнятним. Правильна побудова такого дерева для суперрядка $S_m'="Hello#world\$"$ представлена на рис. ~\ref{fig:png-3}.  

\begin{figure}[h]
\centering
\includegraphics[width=\textwidth]{images/рис. 3.png}
\caption{}
\label{fig:png-3}
\end{figure}


\subsection{Модель багатопотоковості в Node.js}

\subsection{Механізм роботи SharedArrayBuffer}

% =====================================================
% РОЗДІЛ 2
% =====================================================
\section{ПРОЕКТУВАННЯ ТА АРХІТЕКТУРА СИСТЕМИ}

\subsection{Адаптація архітектури DGST}

\subsection{Стратегія управління пам'яттю}

\subsection{Алгоритмічні етапи реалізації}

% =====================================================
% РОЗДІЛ 3
% =====================================================
\section{ПРОГРАМНА РЕАЛІЗАЦІЯ}

\subsection{Середовище розробки та інструменти}

\subsection{Реалізація Майстер-ноди (Main Thread)}

\subsection{Реалізація Виконавчих нод (Workers)}

\subsection{Вирішення проблем синхронізації та безпеки}

% =====================================================
% РОЗДІЛ 4
% =====================================================
\section{ЕКСПЕРИМЕНТАЛЬНЕ ДОСЛІДЖЕННЯ ТА АНАЛІЗ ЕФЕКТИВНОСТІ}

% =====================================================
% ВИСНОВКИ
% =====================================================
\section*{ВИСНОВКИ}
\addcontentsline{toc}{section}{ВИСНОВКИ}

\end{document}
